\documentclass[a4paper]{article}

% Options possibles : 10pt, 11pt, 12pt (taille de la fonte)
%                     oneside, twoside (recto simple, recto-verso)
%                     draft, final (stade de développement)

\usepackage[utf8]{inputenc}   % LaTeX, comprends les accents !
\usepackage[T1]{fontenc}      % Police contenant les caractères français
\usepackage[francais]{babel}  % Placez ici une liste de langues, la
                              % dernière étant la langue principale

\usepackage[a4paper]{geometry}% Réduire les marges
% \pagestyle{headings}        % Pour mettre des entêtes avec les titres
                              % des sections en haut de page

\title{Calcul de couture minimale}           % Les paramètres du titre : titre, auteur, date
\author{Maxime Broy \and Florent Guiotte}
\date{}                       % La date n'est pas requise (la date du
                              % jour de compilation est utilisée en son
			      % absence)

\sloppy                       % Ne pas faire déborder les lignes dans la marge

\begin{document}

\maketitle                    % Faire un titre utilisant les données
                              % passées à \title, \author et \date

\begin{abstract}
L'objectif initial du projet était l'implémentation du papier Scene Completion Using Millions of Photographs [Hays et Efros. 2007]. Ce dernier consiste à remplacer dans une image toute une région par une scène présente dans une autre image provenant d'une collection de photographies. L'algorithme explore cette collection jusqu'à trouver une image avec une sémantique correspondante. On cherche donc à remplacer une région de l'image cible par quelque chose de "plausible". Il s'agit d'une technique difficilement quantifiable puisque le résultat repose sur la perception visuel humaine. 
\end{abstract}

% \tableofcontents              % Table des matières

% \part{Titre}                % Commencer une partie...

\section{Introduction et concept}               % Commencer une section, etc.

Nous avons découpé le travail en plusieurs briques : 

\section{Méthode}         % Section plus petite

Praesent adipiscing nisi id augue consectetuer ultrices. Aenean hendrerit tortor
quis magna condimentum accumsan. In a tortor. Fusce ut augue. Aenean interdum,
metus sit amet mollis tincidunt, nunc sapien viverra sapien, in convallis leo
diam sed nunc. Vivamus semper erat non leo. Vivamus ac ipsum eu velit convallis
blandit. Class aptent taciti sociosqu ad litora torquent per conubia nostra, per
inceptos himenaeos.

\subsection{Grabcut}       % Encore plus petite

Praesent adipiscing nisi id augue consectetuer ultrices. Aenean hendrerit tortor
quis magna condimentum accumsan. In a tortor. Fusce ut augue. Aenean interdum,
metus sit amet mollis tincidunt, nunc sapien viverra sapien, in convallis leo
diam sed nunc. Vivamus semper erat non leo. Vivamus ac ipsum eu velit convallis
blandit. Class aptent taciti sociosqu ad litora torquent per conubia nostra, per
inceptos himenaeos.

\subsection{Calcul d'énergie}

Quisque dolor odio, aliquam quis, placerat sed, hendrerit eu, magna. Cras at
turpis et mi imperdiet lobortis. Nam eu massa et eros congue gravida. Sed
luctus. Nullam sit amet nunc a tellus lacinia tempor. Praesent tincidunt ligula
quis lacus. Nullam sodales, mi sed venenatis egestas, risus turpis dictum elit,
ac egestas augue eros eget erat. Cras faucibus.

\subsection{Calcul d'énergie}

Quisque dolor odio, aliquam quis, placerat sed, hendrerit eu, magna. Cras at
turpis et mi imperdiet lobortis. Nam eu massa et eros congue gravida. Sed
luctus. Nullam sit amet nunc a tellus lacinia tempor. Praesent tincidunt ligula
quis lacus. Nullam sodales, mi sed venenatis egestas, risus turpis dictum elit,
ac egestas augue eros eget erat. Cras faucibus.

\subsection{Calcul des cartes d'énergie cumulée}

Quisque dolor odio, aliquam quis, placerat sed, hendrerit eu, magna. Cras at
turpis et mi imperdiet lobortis. Nam eu massa et eros congue gravida. Sed
luctus. Nullam sit amet nunc a tellus lacinia tempor. Praesent tincidunt ligula
quis lacus. Nullam sodales, mi sed venenatis egestas, risus turpis dictum elit,
ac egestas augue eros eget erat. Cras faucibus.

\subsection{Calcul de la couture}

Quisque dolor odio, aliquam quis, placerat sed, hendrerit eu, magna. Cras at
turpis et mi imperdiet lobortis. Nam eu massa et eros congue gravida. Sed
luctus. Nullam sit amet nunc a tellus lacinia tempor. Praesent tincidunt ligula
quis lacus. Nullam sodales, mi sed venenatis egestas, risus turpis dictum elit,
ac egestas augue eros eget erat. Cras faucibus.

\section{Résultat}

Quisque dolor odio, aliquam quis, placerat sed, hendrerit eu, magna. Cras at
turpis et mi imperdiet lobortis. Nam eu massa et eros congue gravida. Sed
luctus. Nullam sit amet nunc a tellus lacinia tempor. Praesent tincidunt ligula
quis lacus. Nullam sodales, mi sed venenatis egestas, risus turpis dictum elit,
ac egestas augue eros eget erat. Cras faucibus.

\section{Performance}

Quisque dolor odio, aliquam quis, placerat sed, hendrerit eu, magna. Cras at
turpis et mi imperdiet lobortis. Nam eu massa et eros congue gravida. Sed
luctus. Nullam sit amet nunc a tellus lacinia tempor. Praesent tincidunt ligula
quis lacus. Nullam sodales, mi sed venenatis egestas, risus turpis dictum elit,
ac egestas augue eros eget erat. Cras faucibus.

\section{Références}

[1] J. Hays and A. A. Efros.  2007. Scene completion using millions of photographs. ACM Trans. Graph.,

[2] JIA, J., SUN, J., TANG, C.-K., AND SHUM, H.-Y. 2006. Drag and-drop pasting. ACM Trans. Graph.,


\end{document}
